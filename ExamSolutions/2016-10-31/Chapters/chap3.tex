\section{ Multi stage rockets:V(height).alts.Multistage }\label{sec:q3}    
\subsection{a}
The velocity can be expressed in terms of the different $\Delta V$ components that have been given in the question. 

For the initial velcity, 
\begin{equation}
V_\text{ignition} = 0\;\text{m/s}
\end{equation}
For the first stage velocity is just the first $\Delta V$. This includes the gravity losses.
\begin{equation}
V_\text{burnout1} = \Delta V_1\;\text{m/s}
\end{equation}
For the second stage velocity is just the first and second $\Delta V$. This includes the gravity losses.
\begin{equation}
V_\text{burnout2} = (\Delta V_1+\Delta V_2)\;\text{m/s}
\end{equation}
Culmination point is per definition:
\begin{equation}
V_\text{culimination} = 0\;\text{m/s}
\end{equation}
\subsection{b}
The height can also be expressed in terms of altitude differences.
\begin{equation}
h_\text{ignition} = 0\;\text{m}
\end{equation}
For the first stage altitude is just the first $\Delta h$. 
\begin{equation}
h_\text{burnout1} = \Delta h_1\;\text{m}
\end{equation}
For the second stage it is just the first and second $\Delta V$ but now also includes the velocity due to the first stage times the burn time of the second stage. 
\begin{equation}
V_\text{burnout2} = (\Delta h_1+\Delta h_2+\Delta V_1t_{b_2})\;\text{m}
\end{equation}
Culmination point is the altitude up to burnout of stage 2. After that the altitude increase is computed with the energy equation:
\begin{equation}
\frac{1}{2}mV^2=mg_0\Delta h\;\rightarrow\;\Delta h = \frac{(\Delta V_1+\Delta V_2)^2}{2g_0}
\end{equation}
So the total altitude in the culmination point becomes
\begin{equation}
\Delta h_c =  (\Delta h_1+\Delta h_2+\Delta V_1t_{b_2})+\frac{(\Delta V_1+\Delta V_2)^2}{2g_0}
\end{equation}

\subsection{c}
For the velocity, the gravity losses in the coast period has to be included.
For the sake of brevity, the culmination and ignition velocity are omitted. The velocity after the first stage is 
\begin{equation}
V_\text{burnout1} = \Delta V_1\;\text{m/s}
\end{equation}
The velocity after the second stage, is the increase in velocity due to the two stage and includes a term for the gravity losses during the coasting period.
\begin{equation}
V_\text{burnout2} = (\Delta V_1-g_0t_{co}+\Delta V_2)\;\text{m/s}
\end{equation}

For the altitude, the ignition altitude is once again 0. The altitude after the first stage is not different.
\begin{equation}
h_\text{burnout1} = \Delta h_1\;\text{m}
\end{equation}
The altitude after the second stage is different. The velocity at the end of the second stage differs. There is an altitude increase due to the coast.
\begin{equation}
h_\text{burnout2} = \Delta h_1 + \Delta h_2 + \Delta V_1t_{co}-\frac{1}{2}g_0t_{co}^2+(\Delta V_1-g_0t_{co})t_{b_2}
\end{equation}
For the culmination point, the theory remains the same. The altitude already reached plus some term that is computed with the energy equation. The latter is:
\begin{equation}
\Delta h_c = \frac{(\Delta V_1-g_0t_{co}+\Delta V_2)^2}{2g_0}
\end{equation}
\begin{equation}
h_c =  \Delta h_1 + \Delta h_2 + \Delta V_1t_{co}+(\Delta V_1-g_0t_{co})t_{b_2} + \frac{(\Delta V_1-g_0t_{co}+\Delta V_2)^2}{2g_0}
\end{equation}
which can be simplified to
\begin{equation}
h_c = \Delta h_1 +\Delta h_2 + \Delta Vt_{b_2}+\frac{(\Delta V_1+\Delta V_2)^2}{2g_0}-[\Delta V_2 + g_0t_{b_2}]t_{co}
\end{equation}
\begin{equation}
h_c' = h_c -[\Delta V_2 + g_0t_{b_2}]t_{co}
\end{equation}

\subsection{d}
The mass fraction $\Lambda$ becomes smaller for an increase in burnout mass of the stages. This means that the amount $\Delta V$ decreases. This means that both coasting altitude and culmination altitude will be lower. The gravity losses during the coast and to the culmination point are not affected.

\subsection{e}
By subtracting the culmnation altitude $h_c'$ found in (c) with the altitude $h_c$ found in (b), it can be seen what the effect is of the coasting.
\begin{equation}
\Delta h = h_c'-h_c = -[\Delta V_2 + g_0t_{b_2}]t_{co}
\end{equation}
What can be seen is that the $\Delta V$ which includes the gravity losses loses the gravity losses. So the $\Delta V$ is just given by the mass ratio and effective exhaust velocity of the second stage
\begin{equation}
\Delta h_c = -c_{{eff}_2}\ln \Lambda_2t_{co}
\end{equation}
