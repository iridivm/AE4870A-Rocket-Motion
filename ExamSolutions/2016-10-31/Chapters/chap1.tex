\section{ Fundamentals:MC understanding }\label{sec:q1}    
\subsection*{A}
False, the general military aircraft have an initial velocity gain of up to 800m/s. Including other parameters, the answer is correct. Ishim project is mentioned.
\subsection*{B}
True, Propellant mass is not included in the mass fraction.
\subsection*{C}
False, payload is not included
\subsection*{D}
True, inertial range is larger than rotating range since impact point is rotating away.
\subsection*{E}
True, It is always for the maximum shooting range. It uses the formula for $X_ig_0/V_{e_{id}}$. by taking the derivative wrt the launch angle, the optimal launch angle is found. Usually 45 degrees or larger.
\subsection*{F}
False, Due to atmosphere and wind, the accuracy will be lower the longer thorught the lower parts of the atmosphere.
\subsection*{G}
True, The drag of mass ratio decrease if the frontal area remains constant but the mass of the rocket increases. With this ratio, the drag loss is computed.
\subsection*{H}
False, it is used to minimize aerodynamic losses.
\subsection*{I}
True, by applying the expulsed mass flow times the exhaust velocity an extra force is derived.
\subsection*{J}
False, it is shown in the slide for the Unconventional Space Launch. Reasoning is that launching at 0 velocity from a 10 km up does not contribute in the same way as already having an initial velocity of 500+ m/s.